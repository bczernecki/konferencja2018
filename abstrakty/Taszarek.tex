\providecommand{\main}{..} 
\documentclass[\main/boa.tex]{subfiles}

\begin{document}

\section{{Klimatologiczne aspekty burz w~Europie -- porównanie wyników sieci detekcji wyładowań atmosferycznych, obserwacji naziemnych, reanaliz, radiosondaży oraz raportów ESWD (1979-2017)}}

\begin{center}
  {\bf \index[a]{Taszarek Mateusz} Mateusz Taszarek$^{1^\star}$,  \index[a]{Czernecki Bartosz} Bartosz Czernecki$^{1}$, \index[a]{Kolendowicz Leszek} Leszek Kolendowicz$^{1}$ }
\end{center}

\vskip 0.3cm

\begin{affiliations}
\begin{enumerate}
\begin{minipage}{0.915\textwidth}
\centering
\item Uniwersytet im. Adama Mickiewicza w Poznaniu \\ Wydział Nauk Geograficznych i Geologicznych  \\ Instytut Geografii Fizycznej i Kształtowania Środowiska Przyrodniczego\\
Zakład Klimatologii \\[-2pt]
\end{minipage}
\end{enumerate}
$^\star$Adres kontaktowy: \href{mailto:tornado@amu.edu.pl}{\nolinkurl{tornado@amu.edu.pl}}\\
\end{affiliations}

\vskip 0.5cm

%\begin{minipage}{0.915\textwidth}
%\keywords R community; package development
%\end{minipage}

\vskip 0.5cm

Niebezpieczne zjawiska atmosferyczne związane z głęboką konwekcją stwarzają duże zagrożenie dla życia i mienia ludzkiego. Każdego roku w Europie z powodu silnych burz statystycznie ginie około stu osób, a w samej Polsce około dziesięciu. Pomimo zagrożenia jakie te zjawiska generują, niewiele prac naukowych zostało poświęcanych występowaniu burz w skali kontynentu. \\Głównym celem badań jest analiza cykli rocznych dni z burzami oraz silnymi burzami obejmująca lata 1979-2017 (39 lat) i obszar Europy. W pracy porównywane są dane z sieci detekcji wyładowań atmosferycznych EUCLID, obserwacje burz wykonywane na stacjach meteorologicznych (raporty SYNOP), reanaliza ERA-Interim, sondowania atmosferyczne oraz raporty niebezpiecznych zjawisk konwekcyjnych z Europejskiej Bazy Danych ESWD. W pierwszej części pracy określane są wartości progowe parametrów konwekcyjnych odpowiadających za powstawanie burz oraz silnych burz. Następnie przy pomocy reanalizy ERA-Interim oraz sondowań atmosferycznych ustalana jest charakterystyka klimatologiczna ww. stanów atmosfery. Uzyskane wyniki porównywane są z obserwacjami burz prowadzonymi przez człowieka oraz systemy teledetekcji naziemnej. W drugiej części pracy przeprowadzona jest analiza zmian w cyklach rocznych dni z burzami oraz silnymi burzami na przestrzeni ostatnich dekad.\\ Uzyskane wyniki wskazują, że burze występują najczęściej na Półwyspie Apenińskim oraz Bałkańskim, głównie w okresie od maja do sierpnia. W ciągu ostatnich 40 lat zanotowany został wzrost liczby burz w północnej części Europy oraz spadek w południowo-wschodniej.
\end{document}
