\providecommand{\main}{..} 
\documentclass[\main/boa.tex]{subfiles}

\begin{document}

\section{Charakterystyka odbiciowości~w~obrębie układu MCC 11 sierpnia 2017 r.}

\begin{center}
  {\bf \index[a]{Pilorz Wojciech} Wojciech Pilorz$^{1,2\star}$}
\end{center}

\vskip 0.3cm

\begin{affiliations}
\begin{enumerate}
\begin{minipage}{0.915\textwidth}
\centering
\item Uniwersytet Śląski\\Wydział Nauk o Ziemi\\ Katedra Klimatologii\\[-2pt]
\item Stowarzyszenie Skywarn Polska – Polscy Łowcy Burz\\[-2pt]
\end{minipage}
\end{enumerate}
$^\star$Adres kontaktowy: \href{mailto:ks.piasecki@student.uw.edu.pl}{\nolinkurl{ks.piasecki@student.uw.edu.pl}}\\
\end{affiliations}

\vskip 0.5cm

%\begin{minipage}{0.915\textwidth}
%\keywords R community; package development
%\end{minipage}

\vskip 0.5cm

W dniu 11 sierpnia 2017 r., przez Polskę przemieścił się rozległy układ MCC z wbudowaną linią szkwału i mezocyklonami. Długość pasa zniszczeń pozwala zaklasyfikować burzę jako derecho. W wyniku przejścia układu, powstały bardzo rozległe szkody wywołane wiatrami prostoliniowymi. Do European Severe Weather Database (ESWD) zgłoszono 1239 raportów o wystąpieniu groźnych zjawisk, z czego 1182 raporty stanowiły informacje o szkodach wiatrowych a 36 raportów to informacje o dużym gradzie. W~początkowym stadium rozwoju, superkomórka powodowała opady gradu o średnicy do 5,5 cm. Niniejszy referat omawia strukturę odbiciowości~w~obrębie przedmiotowego układu MCC, ze szczególnym uwzględnieniem superkomórki burzowej, która spowodowała największe szkody. W obrębie układu stwierdzono występowanie bardzo rozległego obszaru niskiej odbiciowości~w~miejscu występowania prądu wstępującego na linii szkwału. Oprócz wspomnianej strefy, na czele burzy występował również bardzo wąski obszar o niższej odbiciowości, spowodowany występowaniem~w~tym miejscu mezocyklonu.

\end{document}
