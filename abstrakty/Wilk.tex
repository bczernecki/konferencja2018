\providecommand{\main}{..} 
\documentclass[\main/boa.tex]{subfiles}

\begin{document}

\section{Miejska wyspa ciepła~w~Łodzi podczas fali chłodu zimą 2018 r.}

\begin{center}
  {\bf \index[a]{Wilk Szymon} Szymon Wilk$^{1^\star}$}
\end{center}

\vskip 0.3cm

\begin{affiliations}
\begin{enumerate}
\begin{minipage}{0.915\textwidth}
\centering
\item Uniwersytet Łódzki \\ Wydział Nauk Geograficznych \\ Katedra Meteorologii~i~Klimatologii\\[-2pt]
\end{minipage}
\end{enumerate}
$^\star$Adres kontaktowy: \href{mailto:szymon_wilk@yahoo.com}{\nolinkurl{szymon_wilk@yahoo.com}}\\
\end{affiliations}

\vskip 0.5cm

%\begin{minipage}{0.915\textwidth}
%\keywords R community; package development
%\end{minipage}

\vskip 0.5cm

Miejska wyspa ciepła jest znanym zjawiskiem klimatycznym występującym na terenach o zwartej zabudowie. Charakteryzuje się ona podwyższoną temperaturą powietrza~w~mieście,~w~stosunku do terenów je otaczających. Fenomen ten posiada charakterystyczny rozkład przestrzenny, a także swój cykl dobowy~i~sezonowy związany~z~czynnikami urbanistycznymi oraz meteorologicznymi. Szczególnie dobrze zaznacza się ona zimą podczas sytuacji antycyklonalnych,~z~którymi związane są fale chłodu.

	Celem opracowania jest zaprezentowanie podstawowych charakterystyk statystycznych: przestrzennych~i~czasowych miejskiej wyspy ciepła~w~Łodzi podczas fali chłodu zimą 2018 roku.


\end{document}
