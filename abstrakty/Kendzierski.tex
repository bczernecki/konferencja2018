\providecommand{\main}{..} 
\documentclass[\main/boa.tex]{subfiles}

\begin{document}

\section{Wpływ warunków synoptycznych na wielkość i rodzaj zachmurzenia na Svalbardzie w letnim sezonie badawczym 2016}

\begin{center}
  {\bf  \index[a]{Kendzierski Sebastian} Sebastian Kendzierski$^{1^\star,2}$, \index[a]{Kolendowicz Leszek} Leszek Kolendowicz$^{1}$, \index[a]{Półrolniczak Marek} Marek Półrolniczak$^{1}$ , \index[a]{Szyga-Pluta Katarzyna} Katarzyna Szyga-Pluta$^{1}$,  \index[a]{Laska Kamil} Kamil Laska$^{3}$}
\end{center}

\vskip 0.3cm

\begin{affiliations}
\begin{enumerate}
\begin{minipage}{0.915\textwidth}
\centering
\item Uniwersytet im. Adama Mickiewicza w Poznaniu \\ Wydział Nauk Geograficznych i Geologicznych  \\ Instytut Geografii Fizycznej i Kształtowania Środowiska Przyrodniczego\\
Zakład Klimatologii \\[-2pt]
\item ENEA Trading Sp. z o.o.
\item Masaryk University in Brno \\ Faculty of Science\\ Department of Geography
\end{minipage}
\end{enumerate}
$^\star$Adres kontaktowy: \href{mailto:wrf@amu.edu.pl}{\nolinkurl{wrf@amu.edu.pl}}\\
\end{affiliations}

\vskip 0.5cm

%\begin{minipage}{0.915\textwidth}
%\keywords R community; package development
%\end{minipage}

\vskip 0.5cm

Prezentacja ma na celu przedstawienie wyników opracowania warunków meteorologicznych dla obszaru zachodniej części Zatoki Petuniabukta (Svalbard) w lipcu 
i sierpniu 2016 roku. Podstawą analizy są dane obserwacyjne prowadzone w pobliżu Stacji Polarnej Uniwersytetu im. Adama Mickiewicza w Poznaniu oraz dane obserwacyjne ze stacji meteorologicznej Svalbard-Lufthavn. W~przeprowadzonej analizie dokonano analizy wpływu warunków synoptycznych na wielkość i rodzaj zachmurzenia. Dla parametrów meteorologicznych obliczono podstawowe miary statystyczne oraz przeanalizowano ich uśrednione przebiegi dobowe. Biorąc za podstawę dane z reanaliz meteorologicznych (NCEP/NCAR) dotyczące średniego ciśnienia na poziomie morza (SLP), wysokości geopotencjału 500~hPa oraz temperatury powietrza na wysokości geopotencjalnej 850~hPa (T850) dla wyróżnionych dni skonstruowano mapy kompozytowe sytuacji synoptycznej dla obszaru o współrzędnych 30W-60E, 70N-85N. Uzyskane mapy porównano ze średnią sytuacją synoptyczną dla lipca i sierpnia z okresu 1948-2016 tworząc kompozytowe mapy anomalii wziętych pod uwagę elementów meteorologicznych.

\end{document}
