\providecommand{\main}{..} 
\documentclass[\main/boa.tex]{subfiles}
\begin{document}
\sloppy




\section{Rozmieszczenie winnic w Polsce w powiązaniu z siecią stacji meteorologicznych w aspekcie określenia mezoklimatu korzystnego do uprawy winorośli}

\begin{center}
  {\bf \index[a]{Sękowski Oskar} Oskar Sękowski$^{1^\star}$}
\end{center}

\vskip 0.3cm

\begin{affiliations}
\begin{enumerate}
\begin{minipage}{0.915\textwidth}
\centering
\item Uniwersytet Jagielloński \\ Wydział Geografii i Geologii \\ Katedra Klimatologii\\[-2pt]
\end{minipage}
\end{enumerate}
$^\star$Adres kontaktowy: \href{mailto:oskar.sekowski@doctoral.uj.edu.pl}{\nolinkurl{oskar.sekowski@doctoral.uj.edu.pl}}\\
\end{affiliations}

\vskip 0.5cm

%\begin{minipage}{0.915\textwidth}
%\keywords R community; package development
%\end{minipage}

\vskip 0.5cm

\textbf{Kontekst:} Zakładanie winnicy powinno być oparte o analizę danych ze stacji meteorologicznych, które są zlokalizowane w odległości nie większej niż 30 km (Bosak, 2006).


\textbf{Cel:} Celem opracowania jest określenie, dla których regionów winiarskich w Polsce możliwa jest syntetyczna charakterystyka mezoklimatu, a dzięki temu określenie potencjalnych warunków klimatycznych i meteorologicznych do uprawy winorośli w Polsce.

\textbf{Metody:} Poprzez inwentaryzację na podstawie witryn internetowych została stworzona baza danych o uprawach winorośli w Polsce. Na mapie wykonanej~w~programie ArcGis przedstawiono rozmieszczenie winnic oraz stacji meteorologicznych. Metodą poligonów wyznaczony został zasięg stacji w promieniu 30 km. Podkład mapy stanowi model wysokościowy SRTM.

\textbf{Wyniki:} Zestawienie obejmuje 395 polskich winnic. Mapa przedstawia rozmieszczenie winnic względem stacji meteorologicznych IMGW oraz ukształtowanie powierzchni kraju. Ponad 79\% polskich winnic znajduje się w promieniu 30 km od najbliższej stacji meteorologicznej i większość znajduje się na wysokości do 400 m.n.p.m oraz w dolinach rzek. 

\end{document}
