\providecommand{\main}{..} 
\documentclass[\main/boa.tex]{subfiles}

\begin{document}

\section{Bilans radiacyjny obszarów bagiennych -- badania pilotażowe w Biebrzańskim Parku Narodowym}

\begin{center}
  {\bf \index[a]{Wieczorek Luiza} Luiza Wieczorek$^{1^\star}$}
\end{center}

\vskip 0.3cm

\begin{affiliations}
\begin{enumerate}
\begin{minipage}{0.915\textwidth}
\centering
\item Uniwersytet Łódzki \\ Wydział Nauk Geograficznych \\ Katedra Meteorologii i Klimatologii\\[-2pt]
\end{minipage}
\end{enumerate}
$^\star$Adres kontaktowy: \href{mailto:wieczorek-luiza@o2.pl}{\nolinkurl{wieczorek-luiza@o2.pl}}\\
\end{affiliations}

\vskip 0.5cm

%\begin{minipage}{0.915\textwidth}
%\keywords R community; package development
%\end{minipage}

\vskip 0.5cm

Biebrzański Park Narodowy to największy park narodowy w Polsce. Ochronie podlegają tam unikatowe i cenne z przyrodniczego punktu widzenia tereny bagienne. \\
Stanowisko pomiarowe zlokalizowane jest w bliskiej odległości od niewielkiej rzeki Kopytkówki w Środkowym Basenie Biebrzy (53°35'30,8"N, 22°53'32,4"E, 109 m n.p.m.). Teren wokół stacji porasta roślinność szuwarowa. Danych o bilansie radiacyjnym dostarcza radiometr różnicowy CNR1 – umieszczony na wysokości 2,7 m. \\
Dane wykorzystane w opracowaniu obejmują okres badawczy od listopada 2012 roku do sierpnia 2016 roku.\\
Bilans radiacyjny zaznacza się wyraźnym przebiegiem rocznym z wartościami najniższymi w okresie zimowym, i najwyższymi latem. Dla całego okresu badawczego średnia wartość bilansu wynosi 63,23 W/m$^{2}$. Rozpatrując średnie dobowe wartości bilansu radiacyjnego w poszczególnych miesiącach najwyższe wartości przypadają na czerwiec: 2013 -- 150 W/m$^{2}$, 2015 -- 155 W/m$^{2}$, 2016 -- 158 W/m$^{2}$), najniższe na grudzień: 2013 (-12) W/m$^{2}$, 2014               (-6,9) W/m$^{2}$, 2015 (-9,7) W/m$^{2}$. Wyjątkiem jest rok 2014 gdzie maksymalne wartości dobowe bilans radiacyjny osiągał w lipcu (ok. 153 W/m$^{2}$). Szczegółowo przeanalizowano składniki bilansu promieniowania krótkofalowego i długofalowego, dodatkowo w ujęciu sum energii. 




\end{document}
