\providecommand{\main}{..} 
\documentclass[\main/boa.tex]{subfiles}

\begin{document}

\section{Wpływ węgla organicznego~w~glebie na zmiany klimatyczne i jakość żywności a inicjatywa '4\%' (na przykładzie Torredelcampo~w~prowincji Jaen~w~Hiszpanii)}

\begin{center}
  {\bf \index[a]{Franczyk Wioleta} Wioleta Franczyk$^{1^\star}$}
\end{center}

\vskip 0.3cm

\begin{affiliations}
\begin{enumerate}
\begin{minipage}{0.915\textwidth}
\centering
\item Uniwersytet im. Adama Mickiewicza~w~Poznaniu\\ Wydział Nauk Geograficznych i Geologicznych\\ Instytut Geografii Społeczno-Ekonomicznej i Gospodarki Przestrzennej \\[-2pt]
\end{minipage}
\end{enumerate}
$^\star$Adres kontaktowy: \href{mailto:franczyk.wioleta.geo@o2.pl}{\nolinkurl{franczyk.wioleta.geo@o2.pl}}\\
\end{affiliations}

\vskip 0.5cm

%\begin{minipage}{0.915\textwidth}
%\keywords R community; package development
%\end{minipage}

\vskip 0.5cm

Ziemska biosfera odgrywa bardzo dużą rolę~w~obiegu węgla, gleby natomiast są ważnym jego zbiornikiem oraz odgrywają znaczącą rolę~w~zmianach klimatycznych. Nawet małe zmiany ilościowe zmagazynowanego węglu organicznego~w~glebie (SOCS) mogą mieć duży wpływ na koncentrację atmosferycznego dwutlenku węgla (CO2), co prowadzi do przyspieszenia procesu globalnego ocieplenia. Ważną rolę~w~próbie ograniczenia tego zjawiska, a tym samym zmniejszenia produkcji dwutlenku węgla, odegrało Porozumienie Klimatyczne~w~Paryżu z 2015 roku oraz Inicjatywa 4/1000, mające na uwadze proces zmniejszania CO2~w~atmosferze, kosztem zwiększania zawartości SOCS~w~glebie. Głównym założeniem~w~pracy jest określenie zawartości węgla organicznego~w~glebie~w~różnych warunkach na przykładzie gajów oliwnych~w~Torredelcampo oraz ewentualne prognozy~w~odniesieniu do zmniejszenia emisji gazów cieplarnianych na skutek zastosowania głównych założeń Inicjatywy 4/1000. Zaprezentowane wyniki badań będą obejmowały dwa przedziały czasowe: lata 2002-2005 oraz 2003-2013. Zostaną przedstawione najważniejsze cele 21-szej Konferencji Organizacji Narodów Zjednoczonych z 2015 roku oraz Inicjatywy "4/1000 – gleby dla bezpiecznej żywności i klimatu". Konkluzją poruszanego tematu będzie przedstawienie możliwych rozwiązań~w~celu ograniczenia postępującego globalnego ocieplenia oraz podporządkowywania się założeniom Porozumienia Klimatycznego i Inicjatywy 4/1000.

\end{document}
