\providecommand{\main}{..} 
\documentclass[\main/boa.tex]{subfiles}

\begin{document}

\section{Szkoła, uniwersytet~i~biznes jako węzły sieci uczącej się. Konstrukcja przyszkolnego ogródka meteorologicznego~w~ramach projektu "Łejery! Co to za pogoda?!"}

\begin{center}
  {\bf \index[a]{Sypniewski Jakub} Jakub Sypniewski$^{1^\star}$}
\end{center}

\vskip 0.3cm

\begin{affiliations}
\begin{enumerate}
\begin{minipage}{0.915\textwidth}
\centering
\item Uniwersytet im. Adama Mickiewicza~w~Poznaniu\\ Wydział Nauk Geograficznych~i~Geologicznych\\ Instytut Geoekologii~i~Geoinformacji \\ Pracownia Dydaktyki Geografii~i~Edukacji Ekologicznej\\[-2pt]
\end{minipage}
\end{enumerate}
$^\star$Adres kontaktowy: \href{mailto:jaksyp@amu.edu.pl}{\nolinkurl{jaksyp@amu.edu.pl}}\\
\end{affiliations}

\vskip 0.5cm

%\begin{minipage}{0.915\textwidth}
%\keywords R community; package development
%\end{minipage}

\vskip 0.5cm

Obserwowany współcześnie nacisk, jaki środowisko cyfrowe wywiera na szkołę, stwarzając uczniom atrakcyjną alternatywę rozwoju (Dylak, 2013) sprawia, że dotychczasowy dylemat edukacyjny „wiedzieć jak” (\emph{know-how}), czy też „wiedzieć co” (\emph{know-what}), coraz częściej zastępowany zostaje przez „wiedzieć gdzie”  (know-where), co oznacza, że wiedza o tym, gdzie znaleźć informacje jest ważniejsza niż znajomość informacji (Frankowski 2015).

Konektywizm postrzega wiedzę jako zbiór połączeń sieciowych pomiędzy obiektami, natomiast uczenie się jest~w~tej perspektywie tworzeniem~i~rozwijaniem takich połączeń (Downes, 2017). Realizacja wyznaczonych celów odbywa się poprzez osobiste doświadczenie ucznia, dominuje zatem, zgodnie~z~podziałem D. Klus-Stańskiej (2000) odwołującym się do kryterium samodzielności ucznia, przyswajanie wiedzy „w poszukiwaniu śladu” zamiast „po śladzie”.  W ramach projektu „Łejery! Co to za pogoda?!” prowadzonego przy udziale trzech podmiotów: 1) szkoły (organizatora projektu), 2) Wydziału Nauk Geograficznych~i~Geologicznych (patrona merytorycznego przedsięwzięcia) oraz 3) firmy Enea (sponsora Enea Akademia Talentów) stworzono sieć uczącą się, zgodną~z~założeniami teorii konstruktywizmu. Głównym celem projektu jest aktywizacja~i~integracja społeczności szkolnej (uczniów, nauczycieli, rodziców) oraz partnerów zewnętrznych (UAM, Enea) poprzez pogłębianie wiedzy na temat środowiska geograficznego, jego ochrony oraz wpływu człowieka na zmiany klimatu dzięki stworzeniu profesjonalnego ogródka meteorologicznego oraz realizacji cyklu warsztatów~z~zakresu nauk przyrodniczych~i~meteorologii. Uczniowie poprzez wspólne obserwacje~i~doświadczenia mają możliwość konstruowania własnej wiedzy~i~rozwijania lub doskonalenia umiejętności praktycznych. 

\end{document}
