\providecommand{\main}{..} 
\documentclass[\main/boa.tex]{subfiles}

\begin{document}

\section{Niezwykle zimne i niezwykle ciepłe miesiące w Europie Środkowej \mbox{(1795-2015)}}

\begin{center}
  {\bf \index[a]{Skrzyńska Magdalena} Magdalena Skrzyńska$^{1^\star}$}
\end{center}

\vskip 0.3cm

\begin{affiliations}
\begin{enumerate}
\begin{minipage}{0.915\textwidth}
\centering
\item Uniwersytet Jagielloński\\ Wydział Geografii i Geologii\\ Zakład Klimatologii\\[-2pt]
\end{minipage}
\end{enumerate}
$^\star$Adres kontaktowy: \href{mailto:magdalena_michalik@o2.pl}{\nolinkurl{magdalena_michalik@o2.pl}}\\
\end{affiliations}

\vskip 0.5cm

%\begin{minipage}{0.915\textwidth}
%\keywords R community; package development
%\end{minipage}

\vskip 0.5cm

W prezentacji podjęto zagadnienie występowania niezwykle zimnych i ciepłych miesięcy w Europie Środkowej -- zjawiska rzadkiego, ale wywołującego wiele negatywnych skutków w rolnictwie czy gospodarce. Scharakteryzowano przebieg roczny i wieloletni ich częstości oraz wielkość anomalii w 220-leciu \mbox{1795--2015}.  Na podstawie kryterium statystycznego plus/minus dwa odchylenia standardowe od średniej wieloletniej temperatury wyłoniono niezwykle zimne miesiące (NZM) i niezwykle ciepłe miesiące (NCM) na 3 stacjach meteorologicznych (Berlin, Kraków, Wiedeń). Stwierdzono, że najwięcej anomalnych pod względem termicznym miesięcy wystąpiło w Berlinie (117), a następnie w Krakowie (109) oraz w Wiedniu (103).  Największa anomalia ujemna wystąpiła w grudniu 1829  w Krakowie, a pozostałych miastach w lutym 1929 (Berlin, Wiedeń). Największa anomalia dodatnia wystąpiła w styczniu 1796  w Wiedniu, natomiast w Krakowie i w Berlinie  w styczniu 2007. Wykazano, że NZM występowały do 1970 i od tej pory zaczęła systematycznie wzrastać częstość NCM. Współczesne ocieplenie przejawia się wzrostem liczby NCM (głównie wiosną i latem), a spadkiem NZM (głównie zimą). Oznacza to, że w Europie mamy do czynienia z ociepleniem asymetrycznym, co oznacza wzrost wariancji temperatury powietrza. 

\end{document}
