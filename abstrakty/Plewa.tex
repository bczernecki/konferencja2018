\providecommand{\main}{..} 
\documentclass[\main/boa.tex]{subfiles}
\begin{document}
\sloppy


\section{Okresy hydrologiczne na wybranych jeziorach w~Polsce w~różnych fazach NAO}


\begin{center}
  {\bf \index[a]{Plewa Katarzyna} Katarzyna Plewa$^{1^\star}$}
\end{center}

\vskip 0.3cm

\begin{affiliations}
\begin{enumerate}
\begin{minipage}{0.915\textwidth}
\centering
\item Uniwersytet im. Adama Mickiewicza~w~Poznaniu\\ Wydział Nauk Geograficznych i Geologicznych\\ Instytut Geografii Fizycznej i Kształtowania Środowiska Przyrodniczego \\Zakład Hydrolologii i Gospodarki Wodnej
\end{minipage}
\end{enumerate}
$^\star$Adres kontaktowy: \href{mailto:katarzyna.plewa@amu.edu.pl}{\nolinkurl{katarzyna.plewa@amu.edu.pl}}\\
\end{affiliations}

\vskip 0.5cm

%\begin{minipage}{0.915\textwidth}
%\keywords R community; package development
%\end{minipage}

\vskip 0.5cm

Okresy hydrologiczne są ważnym narzędziem przy określaniu reżimu stanów wody jezior, który może być modyfikowany przez czynniki klimatyczne, cechy morfometryczne jezior, fizjografię zlewni jeziora oraz działalność gospodarczą człowieka. W Polsce na zmienność warunków klimatycznych dość wyraźnie wpływa natężenie Oscylacji Północnoatlantyckiej (NAO), dlatego celem pracy jest ustalenie zmian typu, czasu trwania i sekwencji okresów hydrologicznych na wybranych jeziorach~w~Polsce~w~różnych fazach NAO. Za okres hydrologiczny przyjęto odcinek czasu o jednolitym typie struktury powiązań zachodzących między pentadami z~punktu widzenia zgodności rozkładów występowania stanów wody (Rotnicka 1977). Okresy hydrologiczne wydzielono na podstawie macierzy podobieństw rozkładów stanów wody dla rocznego zbioru pentad. W pracy wykorzystano codzienne wartości stanów wody dla 10 jezior położonych~w~północnej części Polski z lat 1981-2015. Każdy okres opisano następującymi parametrami: termin początku i końca, czas trwania okresu, średni stan wody, współczynnik zmienności, współczynnik stanu wody, współczynnik skośności, amplituda stanów wody.

Przeprowadzone badania potwierdziły, że wpływ Oscylacji Północnoatlantyckiej na reżim stanów wody jezior nie jest silny, ale zauważalny. W zależności od natężenia NAO zmienia się typ, liczba oraz sekwencja okresów hydrologicznych na badanych jeziorach.
\end{document}
