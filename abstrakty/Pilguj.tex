\providecommand{\main}{..} 
\documentclass[\main/boa.tex]{subfiles}

\begin{document}

\section{Trendy indeksów wiatrowych dla wybranych stacji archipelagu Svalbard}

\begin{center}
  {\bf \index[a]{Pilguj Natalia} Natalia Pilguj$^{1^\star}$, \index[a]{Kolendowicz Leszek} Leszek Kolendowicz$^{2}$, \index[a]{Kryza Maciej} Maciej Kryza$^{1}$, \index[a]{Migała Krzysztof} Krzysztof Migała$^{1}$, \index[a]{Czernecki Bartosz} Bartosz Czernecki$^{2}$}
\end{center}

\vskip 0.3cm

\begin{affiliations}
\begin{enumerate}
\begin{minipage}{0.915\textwidth}
\centering
\item Uniwersytet Wrocławski\\ Wydział Nauk o Ziemi i Kształtowania Środowiska\\Instytut Geografii i Rozwoju Regionalnego\\Zakład Klimatologii i Ochrony Atmosfery\\[-2pt]
\item Uniwersytet im. Adama Mickiewicza w Poznaniu \\ Wydział Nauk Geograficznych i Geologicznych  \\ Instytut Geografii Fizycznej i Kształtowania Środowiska Przyrodniczego\\
Zakład Klimatologii \\[-2pt]
\end{minipage}
\end{enumerate}
$^\star$Adres kontaktowy: \href{mailto:natalia.pilguj@uwr.edu.pl}{\nolinkurl{natalia.pilguj@uwr.edu.pl}}\\
\end{affiliations}

\vskip 0.5cm

%\begin{minipage}{0.915\textwidth}
%\keywords R community; package development
%\end{minipage}

\vskip 0.5cm

Badania przedstawiają analizę kierunku i prędkości wiatru dla wybranych stacji archipelagu Svalbard (Bjornoya, Hopen, Ny Alesund) dla lat 1986-2015. Na podstawie wartości dobowych zostały określone indeksy wiatrowe, które zostały poddane dalszym analizom statystycznym. Na wspomniane indeksy składa się suma dni (roczna lub sezonowa), gdzie wartości prędkości bądź kierunku wiatru mieszczą się w określonych przedziałach. Główną część badań stanowi zastosowanie testu Mann-Kendall, który pozwala na wykrycie trendów w wartościach indeksów z kolejnych lat. Statystycznie istotne trendy spadkowe zostały wykryte dla indeksów które uwzględniają niskie (<2 m s$^{-1}$) oraz duże prędkości wiatru (>10 m s$^{-1}$). Trendy wzrostowe odnotowano w obrębie indeksów reprezentujących prędkości wiatru w zakresie 2-10 m s$^{-1}$ wyłącznie dla stacji Hopen i Ny Alesund. Indeksy kierunku wiatru charakteryzują się odmiennymi tendencjami dla Ny Alesund, gdzie obserwowany jest duży wpływ efektów lokalnych, modyfikujących przeważające kierunki napływu mas powietrza. W celu uzasadnienia obserwowanych zmian warunków anemometrych na Svalbardzie, wykonano test Mann-Kendall dla częstości występowania typów cyrkulacji atmosferycznej wg. Niedźwiedzia (2013). Wykazana wzrastająca częstość typów cyklonalnych, pozwala na uzasadnienie trendu wzrostowego umiarkowanych prędkości wiatru oraz spadkowego tych słabszych. 

\end{document}
