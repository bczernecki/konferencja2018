\providecommand{\main}{..} 
\documentclass[\main/boa.tex]{subfiles}
\begin{document}
\sloppy

\section{Strumienie aerozolu morskiego na Morzu Grenlandzkim }

\begin{center}
  {\bf \index[a]{Dziembor Katarzyna} Katarzyna Dziembor$^{1,2^\star}$ }
\end{center}

\vskip 0.3cm

\begin{affiliations}
\begin{enumerate}
\begin{minipage}{0.915\textwidth}
\centering
\item Uniwersytet Gdański\\ Wydział Oceanografii i Geografii\\ Zakład Oceanografii Fizycznej
\item Instytut Oceanologii Polskiej Akademii Nauk \\ Zakład Dynamiki Morza \\ Pracownia Wzajemnego Oddziaływania Morza i Atmosfery\\[-2pt]
\end{minipage}
\end{enumerate}
$^\star$Adres kontaktowy: \href{mailto:katarzyna.dziembor95@gmail.com}{\nolinkurl{katarzyna.dziembor95@gmail.com}}\\
\end{affiliations}

\vskip 0.5cm

%\begin{minipage}{0.915\textwidth}
%\keywords R community; package development
%\end{minipage}

\vskip 0.5cm

Aerozole morskie są emitowane do atmosfery najintensywniej podczas sztormów i wpływają na wiele czynników klimatycznych, takich jak rozpraszanie światła, albedo chmur czy procesy oczyszczania warstwy granicznej z zanieczyszczeń. Są też ważną częścią wymiany ciepła, masy i pędu pomiędzy morzem i atmosferą, i odgrywają znaczącą rolę~w~fizyce chmur, chemii atmosfery i oceanografii. Pomiary i analiza tego zjawiska jest możliwa dzięki wyznaczeniu strumienia aerozolu morskiego (Sea Spray Flux--SSF). Obecnie emisja szacowana jest na 1600–6800 Tg/y z błędem pomiarowym 80\%, co motywuje do rozwijania metodologii dotyczącej badania tego zjawiska na granicy morza i atmosfery. 

Głównym celem prezentacji jest przedstawienie wyników pomiarów SSF przeprowadzanych z pokładu R/V Oceanii, należącej do IO PAN. Dane zostały zebrane podczas corocznego polarnego rejsu AREX~w~lipcu 2017 r~w~obszarze Arktyki Europejskiej. Użyto~w~tym celu zestawu pięciu optycznych liczników cząstek OPC-N2. SSF został oszacowany przy użyciu metody gradientowej, wzięto również pod uwagę warunki meteorologiczne panujące podczas pomiarów.


\end{document}
