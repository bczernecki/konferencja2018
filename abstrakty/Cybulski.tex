\providecommand{\main}{..} 
\documentclass[\main/boa.tex]{subfiles}

\begin{document}

\section{Wykorzystanie analogu łazika marsjańskiego~w~badaniu wybranych parametrów meteorologicznych}

\begin{center}
  {\bf \index[a]{Cybulski Bartłomiej} Bartłomiej Cybulski$^{1^\star}$,  \index[a]{Wieczorek Luiza} Luiza Wieczorek$^{2}$}
\end{center}

\vskip 0.3cm

\begin{affiliations}
\begin{enumerate}
\begin{minipage}{0.915\textwidth}
\centering
\item Politechnika Łódzka\\ Wydział Elektrotechniki, Elektroniki, Informatyki i Automatyki\\ Zakład Sterowania Robotów
\item Uniwersytet Łódzki \\ Wydział Nauk Geograficznych \\ Katedra Meteorologii i Klimatologii\\[-2pt]
\end{minipage}
\end{enumerate}
$^\star$Adres kontaktowy: \href{mailto:bkcybulski@gmail.com}{\nolinkurl{bkcybulski@gmail.com}}\\
\end{affiliations}

\vskip 0.5cm

%\begin{minipage}{0.915\textwidth}
%\keywords R community; package development
%\end{minipage}

\vskip 0.5cm

Eksperyment pomiarowy odbywał się na terenie Marsjańskiej Stacji Badawczej (MDRS) NASA~w~obrębie pustyni~w~stanie Utah. Środowisko naturalne jest tam najbardziej zbliżone do marsjańskich realiów. Badania przeprowadzane były~w~ramach jednej z konkurencji Międzynarodowego Konkursu University Rover Challenge. 

Platformą testową dla omawianych badań był analog łazika marsjańskiego zbudowany przez drużynę Raptors z Politechniki Łódzkiej. Jest to sześciokołowy robot mobilny oparty o zawieszenie typu rocker-bogie. Sercem całego łazika jest komputer pokładowy SB-Rio. Komunikuje się on z poszczególnymi podzespołami platformy za pomocą interfejsu CAN. Za obsługę modułu badawczego (próbnika) odpowiada dedykowana elektronika, którą zarządza 32-bitowy mikrokontroler~z~rodziny STM32F4~z~rdzeniem ARM Cortex-M4. Jego zadaniem jest obsługa sensorów do badania parametrów meteorologicznych atmosfery.

Całość pomiaru trwała 21 minut. Dane zapisywane były z częstotliwością 60 Hz, natomiast~w~pliku wynikowym uśredniono je do 1 Hz. 

Wykorzystano następujące czujniki: czujnik temperatury i wilgotności SHT75, czujnik zapylenia, czujnik prędkości i kierunku wiatru (anemometr czaszowy). 

Średnia wartość temperatury podczas serii pomiarowej wyniosła: 31,6°C, temperatury punktu rosy: 11,9°C, wilgotność powietrza: 30,9\%, średnia prędkość wiatru: 0,96 m/s. 
Uzyskane wyniki skłaniają do rozwoju badań~w~tym zakresie i przeprowadzenia bardziej złożonych i dłuższych serii pomiarowych. 



\end{document}
