\providecommand{\main}{..} 
\documentclass[\main/boa.tex]{subfiles}

\begin{document}

\section{W jaki sposób modyfikujemy pogodę? Przegląd wybranych sposobów rozpraszania mgieł}

\begin{center}
  {\bf \index[a]{Skomorowski Adam} Adam Skomorowski$^{1^\star}$}
\end{center}

\vskip 0.3cm

\begin{affiliations}
\begin{enumerate}
\begin{minipage}{0.915\textwidth}
\centering
\item Uniwersytet Łódzki \\ Wydział Nauk Geograficznych \\ Katedra Meteorologii i Klimatologii\\[-2pt]
\end{minipage}
\end{enumerate}
$^\star$Adres kontaktowy: \href{mailto:adam.skomorowski@gmail.com}{\nolinkurl{adam.skomorowski@gmail.com}}\\
\end{affiliations}

\vskip 0.5cm

%\begin{minipage}{0.915\textwidth}
%\keywords R community; package development
%\end{minipage}

\vskip 0.5cm

Mgła to zjawisko atmosferyczne, które poprzez zmniejszenie widzialności poziomej, w znaczący sposób utrudnia bezpieczne przemieszczanie się. Miejscem szczególnie wrażliwym na pogorszenie przejrzystości atmosfery poniżej 1000m jest lotnisko. Sprawność oraz bezpieczeństwo procedur takich jak: lądowania, startu czy kołowania statku powietrznego podczas mgły są szczególnie utrudnione. Czas poświęcany na podstawowe czynności związane z obsługą samolotów w momencie wystąpienia mgły ulega znacznemu wydłużeniu. Pierwsze systemy mające za zadanie rozpraszać mgłę na lotniskach testowano już podczas II wojny światowej. Rozwój systemów poprawiania widoczności~w~sytuacji uformowania się mgły trwa do dzisiejszego dnia. Z jaką skutecznością potrafimy dziś modyfikować pogodę aby rozproszyć mgłę? W niniejszej prezentacji przedstawiono wybrane sposoby rozpraszania mgły ze szczególnym uwzględnieniem ich wad oraz zalet.
\end{document}
