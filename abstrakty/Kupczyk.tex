\providecommand{\main}{..} 
\documentclass[\main/boa.tex]{subfiles}

\begin{document}

\section{Wizualizacja danych meteorologicznych przy użyciu różnych bibliotek R}

\begin{center}
  {\bf \index[a]{Kupczyk Mariusz} Mariusz Kupczyk$^{1^\star}$}
\end{center}

\vskip 0.3cm

\begin{affiliations}
\begin{enumerate}
\begin{minipage}{0.915\textwidth}
\centering
\item Uniwersytet im. Adama Mickiewicza w Poznaniu\\ Wydział Nauk Geograficznych i Geologicznych
\end{minipage}
\end{enumerate}
$^\star$Adres kontaktowy: \href{mailto:mariuszkupczyk9429@gmail.com	}{\nolinkurl{mariuszkupczyk9429@gmail.com}}\\
\end{affiliations}

\vskip 0.5cm

%\begin{minipage}{0.915\textwidth}
%\keywords R community; package development
%\end{minipage}

\vskip 0.5cm

Opracowanie dotyczy porównania bibliotek języka R, służących do wizualizacji danych synoptycznych. Porównano funkcje: \emph{tmap}, \emph{ggplot}, oraz \emph{plot}. Referat zawiera elementy oceny szybkości wykonywania funkcji, a także możliwości ich wykorzystania w automatyzacji danych. Praca opiera się o samodzielnie przygotowane funkcje testujące, a także wykorzystane powszechnie dostępne, inne tego typu biblioteki. Projekt jest częścią pracy magisterskiej, dotyczącej wizualizacji danych synoptycznych z wykorzystaniem języka programowania R. \\Wszystkie przedstawione w prezentacji funkcje testowane były na systemie Windows 10, wyposażonego w 8GB pamięci RAM, a także dwurdzeniową jednostkę obliczeniową o mocy 2x2.5GHz. 

\end{document}
