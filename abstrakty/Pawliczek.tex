\providecommand{\main}{..} 
\documentclass[\main/boa.tex]{subfiles}

\begin{document}

\section{Indeksy bioklimatyczne w Karkonoszach Zachodnich w latach 1961-2015}

\begin{center}
  {\bf \index[a]{Pawliczek Piotr} Piotr Pawliczek$^{1^\star}$}
\end{center}

\vskip 0.3cm

\begin{affiliations}
\begin{enumerate}
\begin{minipage}{0.915\textwidth}
\centering
\item Uniwersytet Wrocławski\\ Wydział Nauk o Ziemi i Kształtowania Środowiska\\ Zakład Klimatologii i Ochrony Atmosfery\\[-2pt]
\end{minipage}
\end{enumerate}
$^\star$Adres kontaktowy: \href{mailto:piotr.pawliczek@uwr.edu.pl}{\nolinkurl{piotr.pawliczek@uwr.edu.pl}}\\
\end{affiliations}

\vskip 0.5cm

%\begin{minipage}{0.915\textwidth}
%\keywords R community; package development
%\end{minipage}

\vskip 0.5cm

Na podstawie danych pomiarowych ze Szrenicy i Labskiej Boudy z lat 1961--2015, obliczono szereg wskaźników bioklimatycznych. Wraz ze wzrostem średniej temperatury rocznej, wydłuża się także okres wegetacyjny -- ze 141 dni w okresie 1961--1970 do 162 dni w latach 2001--2015. Średnia wartość indeksów kontynentalizmu termicznego Gorczyńskiego i Johanssona-Ringleba nie wykazuje zmian w przebiegu wieloletnim. Pośród indeksów pluwiotermicznych, rosnąca na przestrzeni lat wartość indeksu Ellenberga zbliża się do wartości granicznej dla klimatu subalpejskiego, a wartość indeksu Langa spada z wartości 790 w okresie 1961--1980 do 476 w okresie 2001--2015. Indeks  de Martonne-a plasuje obie stacje w klimacie ekstremalnie wilgotnym. Najbardziej istotną zmianą jest wzrost sumy opadów w półroczu chłodnym, które w latach 2001--2015 mają równy udział w sumie opadu w ujęciu roku hydrologicznego.

\end{document}
