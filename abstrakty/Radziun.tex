\providecommand{\main}{..} 
\documentclass[\main/boa.tex]{subfiles}

\begin{document}

\section{Zmienność opadów atmosferycznych~w~okresie 1961-2015~w~centralnej części Polski}

\begin{center}
  {\bf \index[a]{Radziun Wojciech} Wojciech Radziun$^{1^\star}$}
\end{center}

\vskip 0.3cm

\begin{affiliations}
\begin{enumerate}
\begin{minipage}{0.915\textwidth}
\centering
\item Uniwersytet Łódzki \\ Wydział Nauk Geograficznych \\ Katedra Meteorologii i Klimatologii\\[-2pt]
\end{minipage}
\end{enumerate}
$^\star$Adres kontaktowy: \href{mailto:wojtekradziun@tlen.pl}{\nolinkurl{wojtekradziun@tlen.pl}}\\
\end{affiliations}

\vskip 0.5cm

%\begin{minipage}{0.915\textwidth}
%\keywords R community; package development
%\end{minipage}

\vskip 0.5cm

Celem opracowania jest analiza zmienności opadów~w~centralnej części Polski. Analiza opiera            się na danych dobowych pozyskanych z IMGW obejmujących lata 1961-2015. Zmienność opadów określono~w~oparciu o odchylenia  sum rocznych i sezonowych opadów atmosferycznych od średniej wieloletniej. Tendencję spadkową opadów atmosferycznych (0,5~mm/rok) stwierdzono dla sum rocznych, jesiennych i letnich, natomiast wzrostową wiosną (0,2~mm/rok) i zimą (0,1~mm/rok). Obserwowane tendencje sezonowych sum opadów nie są istotne statystycznie. Jedynie~w~listopadzie stwierdzono statystycznie istotny trend. Najbardziej wilgotnym rokiem 210~mm ponad normę okazał się 2010~r. Największy niedobór o 171~mm poniżej średniej wieloletniej stwierdzono~w~2015~r. Odchylenie standardowe jest największe latem i sięga 58~mm, a zimą najmniejsze i wynosi 30~mm.


\end{document}
