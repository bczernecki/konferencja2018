\providecommand{\main}{..} 
\documentclass[\main/boa.tex]{subfiles}

\begin{document}

\section{Superkomórki burzowe w południowej~i~zachodniej Polsce dnia 07.07.2017 – analiza synoptyczna}

\begin{center}
  {\bf \index[a]{Poręba Szymon} Szymon Poręba$^{1,2^\star}$}
\end{center}

\vskip 0.3cm

\begin{affiliations}
\begin{enumerate}
\begin{minipage}{0.915\textwidth}
\centering
\item Uniwersytet Jagielloński\\ Instytut Geografii~i~Gospodarki Przestrzennej\\ Zakład Klimatologii
\\[-2pt]
\item Instytut Meteorologii~i~Gospodarki Wodnej -- Państwowy Instytut Badawczy \\[-2pt]
\end{minipage}
\end{enumerate}
$^\star$Adres kontaktowy: \href{mailto:szymon.poreba@doctoral.uj.edu.pl}{\nolinkurl{szymon.poreba@doctoral.uj.edu.pl}}\\
\end{affiliations}

\vskip 0.5cm

%\begin{minipage}{0.915\textwidth}
%\keywords R community; package development
%\end{minipage}

\vskip 0.5cm

7 lipca 2017 roku w południowej~i~zachodniej Polsce wystąpiły liczne superkomórki burzowe. W rejonie Raciborza zaobserwowano trąbę powietrzną o szacowanej sile F2 w skali Fujity.  W wielu miejscach wystąpiły opady gradu, którego rozmiary osiągnęły nawet 5 cm, a porywy wiatru spowodowały liczne zniszczenia. Analiza danych radarowych pozwoliła stwierdzić klasyczny charakter superkomórek~z~wyraźnie wykształconymi elementami echa radarowego. 

Tego dnia obszar południowo-zachodniej Polski znajdował się w obrębie zatoki niskiego ciśnienia wraz~z~ciepłym frontem atmosferycznym, związanym~z~niżem znad środkowych Niemiec. Wschodnia część kraju była pod wpływem  klina wysokiego ciśnienia. W górnej troposferze występował znaczny, zachodni~i~północno-zachodni przepływ powietrza. Taki rozkład baryczny przyczynił się do wystąpienia wyraźnego skrętu wiatru, zwłaszcza w dolnej troposferze. 

W opracowaniu wykorzystano dane ze stacji synoptycznych, raporty~z~bazy ESWD oraz pomiary odbiciowości radarowej~i~widma wiatru dopplerowskiego~z~różnych elewacji.

Celem opracowania było wyszczególnienie charakterystycznych cech superkomórek burzowych za pomocą danych radarowych oraz określenie warunków synoptycznych sprzyjających tego typu burzom. Podjęto również próbę oceny wskaźników konwekcji najlepiej prognozujących superkomórki burzowe.


\end{document}
