\providecommand{\main}{..} 
\documentclass[\main/boa.tex]{subfiles}

\begin{document}
\sloppy
\section{Ocena jakości powietrza~w~Miechowie}

\begin{center}
  {\bf \index[a]{Bielecki Rafał} Rafał Bielecki$^{1^\star}$}
\end{center}

\vskip 0.3cm

\begin{affiliations}
\begin{enumerate}
\begin{minipage}{0.915\textwidth}
\centering
\item Uniwersytet Pedagogiczny~w~Krakowie \\ Wydział Geograficzno-Biologiczny  \\ Instytut Geografii \\
Zakład Ekorozwoju i Kształtowania Środowiska Geograficznego \\[-2pt]
\end{minipage}
\end{enumerate}
$^\star$Adres kontaktowy: \href{mailto:rafal.bielecki1@op.pl}{\nolinkurl{rafal.bielecki1@op.pl}}\\
\end{affiliations}

\vskip 0.5cm

%\begin{minipage}{0.915\textwidth}
%\keywords R community; package development
%\end{minipage}

\vskip 0.5cm

Zanieczyszczenie powietrza pyłem zawieszonym stanowi poważny problem miast oraz aglomeracji miejsko-przemysłowych. Źródłem pyłowych i gazowych zanieczyszczeń powietrza na obszarach miejskich są głównie emisje antropogeniczne pochodzące z sektora komunalno-bytowego, przemysłu oraz środków transportu. Jak wskazują liczne badania epidemiologiczne, zanieczyszczenia atmosferyczne, zwłaszcza pyły drobne o średnicach aerodynamicznych cząstek poniżej 2,5 µm, stanowią zagrożenie dla zdrowia osób długotrwale przebywających~w~warunkach występowania ponadnormatywnych stężeń pyłu.

Celem pracy jest ocena zanieczyszczenia powietrza pyłem zawieszonym PM\textsubscript{10} oraz PM\textsubscript{2,5} ~w~odniesieniu do obowiązujących~w~Polsce norm jakości powietrza. Pomiary były prowadzone~w~okresie 2017 roku i swoim zasięgiem obejmowały Miechów. Miasto położone jest~w~województwie małopolskim przy międzynarodowej trasie nr 7 (E77),~w~odległości około 45 kilometrów od Krakowa i 80 kilometrów od Kielc. Ma powierzchnię 15,49 km² oraz zamieszkuje 11 722 osób przy gęstości zaludnienia 756,7 os./km². Na terenie miasta nie ma zakładów przemysłowych które mogłyby mieć wpływ na złą jakość powietrza. Pomimo tego Miechów znajduje się~w~czołówce rankingu polskich miast z najbardziej zanieczyszczonym powietrzem. Do głównych czynników wpływających na zanieczyszczenie powietrza należy położenie geograficzne. Miasto położone jest~w~centralnej części Wyżyny Miechowskiej,~w~dolinie potoku Miechówki, lewego dopływu Cichej. Ukształtowanie powierzchni utrudnia wentylację miasta, przez co spowija je smog, szczególnie~w~sezonie grzewczym.

Dnia 10 lutego 2017 roku, firma Airly zainstalowała~w~Miechowie sześć sensorów jakości powietrza. Autor na podstawie danych przekazywanych z sensorów rozmieszczonych~w~różnych lokalizacjach miasta jako pierwszy dokonał analizy jakości powietrza~w~Miechowie. Określił dynamikę zmian. Wskazał na główne przyczyny oraz miejsca najbardziej narażone na podwyższą zawartość zanieczyszczeń powietrza. Na podstawie danych - pył zawieszony PM\textsubscript{10}, PM\textsubscript{2,5}, temperatura powietrza, wilgotność oraz ciśnienie dokonał korelacji jakości powietrza. Opracowanie posłuży do uświadomienia mieszkańców Miechowa o jakości powietrza~w~mieście.


\end{document}
