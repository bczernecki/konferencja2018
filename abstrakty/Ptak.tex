\providecommand{\main}{..} 
\documentclass[\main/boa.tex]{subfiles}

\begin{document}

\section{Wpływ wybranych makroskalowych typów cyrkulacji atmosfery na zmiany temperatury wody jezior w Polsce}

\begin{center}
  {\bf \index[a]{Ptak Mariusz} Matriusz Ptak$^{1^\star}$,  \index[a]{Tomczyk Arkadiusz} Arkadiusz Tomczyk$^{2}$, \index[a]{Wrzesiński Dariusz} Dariusz Wrzesiński$^{1}$ }
\end{center}

\vskip 0.3cm

\begin{affiliations}
\begin{enumerate}
\begin{minipage}{0.915\textwidth}
\centering
\item Uniwersytet im. Adama Mickiewicza~w~Poznaniu\\ Wydział Nauk Geograficznych i Geologicznych\\ Instytut Geografii Fizycznej i Kształtowania Środowiska Przyrodniczego \\Zakład Hydrolologii i Gospodarki Wodnej
\item Uniwersytet im. Adama Mickiewicza~w~Poznaniu \\ Wydział Nauk Geograficznych~i~Geologicznych  \\ Instytut Geografii Fizycznej~i~Kształtowania Środowiska Przyrodniczego\\
Zakład Klimatologii \\[-2pt]
\end{minipage}
\end{enumerate}
$^\star$Adres kontaktowy: \href{mailto:mariusz.ptak@amu.edu.pl}{\nolinkurl{mariusz.ptak@amu.edu.pl}}\\
\end{affiliations}

\vskip 0.5cm

%\begin{minipage}{0.915\textwidth}
%\keywords R community; package development
%\end{minipage}

\vskip 0.5cm

Celem pracy jest określenie wpływu wybranych makroskalowych typów cyrkulacji atmosfery: Oscylacja Arktyczna (AO), typ wschodnioatlantycki (EA), typ skandynawski (SCAND) oraz typ wschodnioeuropejski (EAWR) na zmiany temperatury powietrza i wody jezior w Polsce. W ujęciu miesięcznym zbadano związki korelacyjne indeksów typów cyrkulacji z temperaturą powietrza~i~wody jezior. Określono także odchylenia wartości tych elementów w różnych fazach badanych typów cyrkulacji od wartości przeciętnych z lat 1971-2015. Badania wykazały, że wpływ ten jest zróżnicowany. Największe oddziaływanie na temperaturę wody zaobserwowano zimą, kiedy szczególnie widoczny wpływ miała cyrkulacja AO. Odchylenia temperatury wody w stosunku  do średnich z analizowanego wielolecia generalnie oscylowały na poziomie 1,0°C. Przedstawione badania wskazują na złożoność procesów decydujących o zmianach temperatury wód  jezior, na przebieg których wpływa wiele czynników; które poza makroskalową cyrkulacją atmosfery zarówno obejmują te o  charakterze lokalnym czy indywidulanym poszczególnych jezior. 

\end{document}
