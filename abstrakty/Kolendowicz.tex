\providecommand{\main}{..} 
\documentclass[\main/boa.tex]{subfiles}
\begin{document}
\sloppy


\section{Zjawisko miejskiej wyspy ciepła i jeziora chłodu na obszarze Poznania}

\begin{center}
  {\bf \index[a]{Kolendowicz Leszek} Leszek Kolendowicz$^{1^\star}$, \index[a]{Półrolniczak Marek} Marek Półrolniczak$^{1}$ , \index[a]{Czernecki Bartosz} Bartosz Czernecki$^{1}$,  \index[a]{Tomczyk Arkadiusz} Arkadiusz Tomczyk$^{1}$}
\end{center}

\vskip 0.3cm

\begin{affiliations}
\begin{enumerate}
\begin{minipage}{0.915\textwidth}
\centering
\item Uniwersytet im. Adama Mickiewicza~w~Poznaniu \\ Wydział Nauk Geograficznych i Geologicznych  \\ Instytut Geografii Fizycznej i Kształtowania Środowiska Przyrodniczego\\
Zakład Klimatologii \\[-2pt]
\end{minipage}
\end{enumerate}
$^\star$Adres kontaktowy: \href{mailto:leszko@amu.edu.pl}{\nolinkurl{leszko@amu.edu.pl}}\\
\end{affiliations}

\vskip 0.5cm

%\begin{minipage}{0.915\textwidth}
%\keywords R community; package development
%\end{minipage}

\vskip 0.5cm

Prezentowane rezultaty badań opierają się na danych zebranych z 8 rejestratorów sieci pomiarowej Zakładu Klimatologii z obszaru Poznania z okresu od kwietnia 2008 do września 2013 roku, danych ze stacji IMGW-PIB Poznań Ławica oraz na obrazach satelitarnych (Landsat 5 oraz Landsat 8) również z wymienionego okresu. Wykorzystując zdjęcia satelitarne zbadano strukturę pola temperatury powierzchni czynnej (LST)~w~obszarze administracyjnym miasta. Biorąc pod uwagę wyniki z sieci pomiarowej na obszarze miasta dokonano charakterystyki przebiegu dobowego i rocznego miejskiej wyspy ciepłą (UHI) oraz jeziora chłodu (UCL)~w~poszczególnych punktach pomiarowych oraz przeliczono LST na temperaturę na wysokości 2 m nad powierzchnią terenu. Z~kolei na podstawie analizy różnic temperatury powietrza pomiędzy centrum miasta a stacją referencyjna Poznań Ławica określono natężenie maksymalne i średnie zarówno miejskiej wyspy ciepła jak i jeziora chłodu oraz ich przebieg dobowy i roczny. Biorąc pod uwagę typy cyrkulacji atmosfery Niedźwiedzia (2011) zbadano wpływ cyrkulacji atmosfery na częstość występowania obu zjawisk. Określono również średnie sytuacje synoptyczne towarzyszące ich ekstremalnym natężeniom. Rezultaty badań pola temperatury na obszarze Poznania zostały poprzedzone prezentacją poznańskiej serii średniej miesięcznej temperatury powietrza z lat 1848-2016 oraz jej homogeniczności. 

\end{document}
