\providecommand{\main}{..} 
\documentclass[\main/boa.tex]{subfiles}
\begin{document}
\sloppy


\section{Aktywność fotosyntezy Miskanta olbrzymiego \emph{(Miscanthus sinensis gigantea)}~w~okresie wegetacji}

\begin{center}
  {\bf \index[a]{Swider Kryspin} Kryspin \'Swider$^{1^\star}$}
\end{center}

\vskip 0.3cm

\begin{affiliations}
\begin{enumerate}
\begin{minipage}{0.915\textwidth}
\centering
\item  Uniwersytet Przyrodniczy we Wrocławiu\\ Wydział Inżynierii Kształtowania Środowiska~i~Geodezji\\ Instytut Kształtowania~i~Ochrony Środowiska
\end{minipage}
\end{enumerate}
$^\star$Adres kontaktowy: \href{mailto:kryspin.swider@upwr.edu.pl}{\nolinkurl{kryspin.swider@upwr.edu.pl}}\\
\end{affiliations}

\vskip 0.5cm

%\begin{minipage}{0.915\textwidth}
%\keywords R community; package development
%\end{minipage}

\vskip 0.5cm

W Polsce, jak~i~również~w~innych krajach Unii Europejskiej, rośnie zainteresowanie uprawą roślin energetycznych. Czynniki meteorologiczne takie jak: wysokość opadów atmosferycznych, światło, wahania wartości temperatury mają duży wpływ na wzrost~i~rozwój roślin~w~okresie wegetacji. W~szczególności,~w~warunkach klimatycznych Polski, rozkład~w~czasie~i~przestrzeni opadów stanowi główny element~w~zaopatrywaniu gleby~w~wodę. Dostępność wody dla roślin ma istotny wpływ na proces fotosyntezy, gdyż prawidłowe uwodnienie komórek powoduje otwarcie aparatów szparkowych, przez co przyczynia się do polepszenia wymian gazowej. Występowanie okresów suszy może doprowadzić do niedostatecznej ilości wody~w~glebie dostępnej dla roślin,~a~tym samym zahamować wzrost lub nawet doprowadzić do ich obumarcia.

Celem referatu jest ocena aktywności fotosyntetycznej miskanta olbrzymiego (Miscanthus sinensis gigantea)~w~dwóch wariantach dostępności do wody gruntowej. Uprawa ekstensywna miskanta olbrzymiego prowadzona jest na terenie Wydziałowego Obserwatorium Agro~i~Hydrometeorologii, Wydziału Inżynierii Kształtowania Środowiska~i~Geodezji, Uniwersytetu Przyrodniczego we~Wrocławiu. Pomiary były prowadzone~w~okresie od czerwca do października 2012 roku za pomocą urządzenia CI-340 Hand-Held Photosynthesis System (CID, Inc.USA). 
\end{document}
