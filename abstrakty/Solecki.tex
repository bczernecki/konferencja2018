\providecommand{\main}{..} 
\documentclass[\main/boa.tex]{subfiles}

\begin{document}

\section{Zjawisko \emph{downburst} w Polsce w dniu 14 VII 2011 r.}

\begin{center}
  {\bf \index[a]{Solecki Mateusz} Mateusz Solecki$^{1^\star}$, \index[a]{Popławska Joanna} Joanna Popławska$^{1}$}
\end{center}

\vskip 0.3cm

\begin{affiliations}
\begin{enumerate}
\begin{minipage}{0.915\textwidth}
\centering
\item Uniwersytet Warszawski\\ Wydział Geografii i Studiów Regionalnych\\ Zakład Klimatologii\\[-2pt]
\end{minipage}
\end{enumerate}
$^\star$Adres kontaktowy: \href{mailto:mateusz.solecki@student.uw.edu.pl}{\nolinkurl{mateusz.solecki@student.uw.edu.pl}}\\
\end{affiliations}

\vskip 0.5cm

%\begin{minipage}{0.915\textwidth}
%\keywords R community; package development
%\end{minipage}

\vskip 0.5cm

\emph{Downburst} to zjawisko silnego prądu zstępującego, który uderza o grunt z ogromną siłą, a następnie rozchodzi się w różnych kierunkach. Stanowi duże zagrożenie dla lotnictwa. Pierwszy udokumentowany przypadek tego groźnego wiatru prostoliniowego wystąpił 
w USA w 1975 r. i~był bezpośrednią przyczyną katastrofy samolotu pasażerskiego, który podchodził do lądowania na lotnisku JFK w Nowym Jorku. \\
Rozpoczęte wówczas badania wykazały, że nie był to odosobniony przypadek, a tego typu zagrożenie występuje też w innych regionach świata.\\
Celem niniejszego opracowania jest określenie warunków meteorologicznych występujących podczas przejścia zjawiska downburst w Polsce w dniu 14 VII 2011 r. W pracy przedstawiono schemat tworzenia się, przebiegu i zaniku tego zjawiska.  Do analizy wykorzystano mapy synoptyczne, obrazy satelitarne i zdjęcia z miejsca zdarzenia. Posłużono się też diagramami aerologicznymi, na podstawie których obliczono wskaźniki: konwekcyjne, uskoku wiatru i złożone.\\
Wykazano, że opisywany przypadek miał miejsce w warunkach silnej chwiejności atmosfery, podczas największej aktywności układów konwekcyjnych. Największym znaczeniem w detekcji tego zjawiska odznaczały się wskaźniki CAPE, dCAPE, CIN i DLS. 

\end{document}
