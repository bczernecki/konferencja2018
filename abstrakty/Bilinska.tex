\providecommand{\main}{..} 
\documentclass[\main/boa.tex]{subfiles}

\begin{document}

\section{Wykorzystanie modelu HYSPLIT do oznaczania obszarów źródłowych pyłku Ambrozji we Wrocławiu}

\begin{center}
  {\bf \index[a]{Bilińska Daria} Daria Bilińska$^{1^\star}$, \index[a]{Ambelas Skjøth Carlsen} Carsten Ambelas Skjøth$^{2}$, \index[a]{Werner Małgorzata} Małgorzata Werner$^{1}$, \index[a]{Kryza Maciej} Maciej Kryza$^{1}$, \index[a]{Malkiewicz Małgorzata} Małgorzata Malkiewicz$^{1}$, \index[a]{Drzeniecka-Osiadacz Anetta} Anetta Drzeniecka-Osiadacz$^{1}$}
\end{center}

\vskip 0.3cm

\begin{affiliations}
\begin{enumerate}
\begin{minipage}{0.915\textwidth}
\centering
\item Uniwersytet Wrocławski\\ Wydział Nauk o Ziemi i Kształtowania Środowiska\\Zakład Klimatologii i Ochrony Atmosfery\\[-2pt]
\item Zakladam, ze Carsten nie ma tej samej afiliacji -- TODO! \\[-2pt]
\end{minipage}
\end{enumerate}
$^\star$Adres kontaktowy: \href{mailto:daria.bilinska2@uwr.edu.pl}{\nolinkurl{daria.bilinska2@uwr.edu.pl}}\\
\end{affiliations}

\vskip 0.5cm

%\begin{minipage}{0.915\textwidth}
%\keywords R community; package development
%\end{minipage}

\vskip 0.5cm

Celem tej pracy było zbadanie relacji pomiędzy napływającymi masami powietrza a stężeniami pyłku ambrozji we Wrocławiu dla lat 2005-2014, a także sprawdzenie, czy wysokie stężenia pyłku ambrozji mogą być powiązane ze znanymi obszarami występowania ambrozji, takimi jak Kotlina Panońska, Włochy czy Ukraina.\\
Dobowe stężenia pyłku obejmowały lata 2005 - 2014. Pomiary były prowadzone we Wrocławiu, na terenie Instytutu Nauk Geologicznych, z użyciem 7 dniowej, objętościowej pułapki Burkard’a. \\
Obliczenia trajektorii wstecznych przeprowadzono z modelem HYSPLIT. Został on uruchomiony dla dwóch miesięcy (sierpień i wrzesień) dla lat 2005 - 2014, w oparciu o dane meteorologiczne GDAS.\\
Na podstawie tej pracy, potwierdzono istnienie rozpoznanych wcześniej centrów występowania Ambrozji takich jak Kotlina Panońska i Ukraina. Obszary Kotliny Panońskiej i Ukrainy powodują epizody występowania pyłku ambrozji w powietrzu nad południowo-zachodnią Polską.


\end{document}
