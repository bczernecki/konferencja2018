\providecommand{\main}{..} 
\documentclass[\main/boa.tex]{subfiles}

\begin{document}

\section{Duże zmiany średniej dobowej  temperatury powietrza z dnia na dzień~w~{Tatrach}}

\begin{center}
  {\bf \index[a]{Guzik Izabela} Izabela Guzik$^{1^\star}$}
\end{center}

\vskip 0.3cm

\begin{affiliations}
\begin{enumerate}
\begin{minipage}{0.915\textwidth}
\centering
\item Uniwersytet Jagielloński \\Wydział Geografii i Geologii\\ Zakład Klimatologii \\[-2pt]
\end{minipage}
\end{enumerate}
$^\star$Adres kontaktowy: \href{mailto:iza.guzik@doctoral.uj.edu.pl}{\nolinkurl{iza.guzik@doctoral.uj.edu.pl}}\\
\end{affiliations}

\vskip 0.5cm

%\begin{minipage}{0.915\textwidth}
%\keywords R community; package development
%\end{minipage}

\vskip 0.5cm

Zmiany temperatury powietrza z dnia na dzień są istotne nie tylko z klimatycznego punktu widzenia, ale mają także znaczenie praktyczne. Duże spadki lub wzrosty temperatury~w~krótkim okresie czasu mogą powodować poważne skutki dla środowiska i różnych gałęzi gospodarki ale przede wszystkim wpływają na samopoczucie człowieka.

W badaniach m. in. Kossowskiej--Cezak (1982) oraz Fortuniaka i in. (1997) stacje położone~w~obszarach górskich charakteryzują się szczególnie dużymi zmianami temperatury, co jest jednym z czynników decydujących o bodźcowości klimatu tego obszaru. Kotliny cechują się dużą zmiennością temperatury minimalnej,~a~Kasprowy Wierch~i~Śnieżka to jedyne stacje~w~Polsce na których przeważają duże spadki tej temperatury nad wzrostami.

Celem prezentacji jest charakterystyka dużych zmian (większych ±6°C) temperatury średniej dobowej powietrza z dnia na dzień na Kasprowym Wierchu oraz~w~Zakopanem~w~latach 1951-2015. Jest to także próba oceny wpływu wysokości względnej oraz formy terenu na występowanie takich zmian. Uzyskane wyniki badań odniesiono do Krakowa czyli stacji położonej poza obszarem górskim.~w~przypadku zmian największych na tych stacjach określono także sytuację synoptyczną~w~tych dniach.

\end{document}
